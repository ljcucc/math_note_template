\usepackage{graphicx}     % 确保加载 graphicx 宏包
\usepackage{xeCJK}
\usepackage[dvipsnames]{xcolor}

\graphicspath{ {./assets/} }

\setCJKmainfont{LXGW WenKai Mono TC}
\setmainfont{LXGW WenKai Mono TC}

\newcommand{\readerName}{Wuffels Wolf}
\newcommand{\teacherName}{Dr.Drone}

% 定义主宏命令 (保持不变)
\newcommand{\imageAndText}[4]{%
  \par\noindent
  \begin{minipage}[c]{#1}%
    #2% <--- 图片或图片内容放这里
  \end{minipage}%
  \hspace{#3}%
  \begin{minipage}[c]{\dimexpr\linewidth-#1-#3\relax}%
    #4% <--- 文本放这里
  \end{minipage}%
  \par
}

\newcommand{\textAndImage}[4]{%
  \par\noindent
  \begin{minipage}[c]{\dimexpr\linewidth-#1-#3\relax}%
    \begin{flushright}
    #4% <--- 文本放这里
    \end{flushright}
  \end{minipage}%
  \hspace{#3}%
  \begin{minipage}[c]{#1}%
    #2% <--- 图片或图片内容放这里
  \end{minipage}%
  \par
}

\newcommand{\readerConfused}[1]{
  \textAndImage
    {0.25\textwidth} % 左侧区域宽度
    {%
      % 这是 figure 环境里面的内容,但没有 figure 环境本身
      \centering % 确保图片在左侧区域居中
      \includegraphics[width=\linewidth]{a-confused.png} % 使用 \linewidth 让图片宽度充满左侧区域
      \vspace{0.5em} % 图片和标题之间留一点垂直空间
      \centering % 确保标题文本居中
      \small % 使用小字体作为标题
      {\textcolor{YellowOrange}\readerName}
      % 注意:这里没有 \caption{...},这只是普通的文本
      % 也不能使用 \label{...} 来引用
    } % 左侧区域内容 (图片 + 手动标题)
    {1em} % 图片和文本之间的间距
    {\textcolor{YellowOrange}{#1}}
}

\newcommand{\readerEureka}[1]{
  \textAndImage
    {0.25\textwidth} % 左侧区域宽度
    {%
      \centering
      \includegraphics[width=\linewidth]{a-eureka.png}
      \vspace{0.5em}
      \centering
      \small
      {\textcolor{YellowOrange}\readerName}
    }
    {1em}
    {\textcolor{YellowOrange}{#1}}
}

\newcommand{\readerQuestioning}[1]{
  \textAndImage
    {0.25\textwidth}
    {
      \centering  \includegraphics[width=\linewidth]{a-questioning.png}
      \vspace{0.5em}
      \centering \small {\textcolor{YellowOrange}\readerName}
    }
    {1em}
    {\textcolor{YellowOrange}{#1}}
}

\newcommand{\readerExcitement}[1]{
  \textAndImage
    {0.25\textwidth}
    {
      \centering  \includegraphics[width=\linewidth]{a-excitement.png}
      \vspace{0.5em}
      \centering \small {\textcolor{YellowOrange}\readerName}
    }
    {1em}
    {\textcolor{YellowOrange}{#1}}
}


\newcommand{\teacherConfident}[1]{
  \imageAndText
  {0.25\textwidth}
  {
    \centering
    \includegraphics[width=\linewidth]{b-confident.png}
    \vspace{0.5em}
    \centering
    \small
    {\textcolor{MidnightBlue}\teacherName}
  }
  {1em}
  {\textcolor{MidnightBlue}{#1}}
}

\newcommand{\teacherTalking}[1]{
  \imageAndText
    {0.25\textwidth}
    {
      \centering
      \includegraphics[width=\linewidth]{b-talking.png}
      \vspace{0.5em}
      \centering
      \small
      {\textcolor{MidnightBlue}\teacherName}
    }
    {1em}
    {\textcolor{MidnightBlue}{#1}}
}

\newcommand{\teacherPause}[1]{
  \imageAndText
    {0.25\textwidth}
    {
      \centering
      \includegraphics[width=\linewidth]{b-pause.png}
      \vspace{0.5em}
      \centering
      \small {\textcolor{MidnightBlue}\teacherName}
    }
    {1em}
    {\textcolor{MidnightBlue}{#1}}
}
