\documentclass[class=article, crop=false, margin=2cm]{standalone}
\usepackage[subpreambles=true]{standalone}

\usepackage{xcolor}
\usepackage{graphicx}     % 确保加载 graphicx 宏包
\usepackage{lipsum}
% \usepackage[margin=2cm]{geometry}
\usepackage{xeCJK}
\usepackage{amsmath}

\begin{document}

\break
\section{線性代數的第一站:什麼是線性組合?}

\readerConfused{「線性組合」聽起來好抽象,我不太能想像它到底是什麼。是像組合數學那樣的意思嗎?}

\smallerVerticalGap
\smallerVerticalGap

\teacherTalking{這是個好問題!雖然「組合」這個詞有點像,但「線性組合」其實是在講「用一堆向量加起來並且乘上係數」的概念。}
\smallerVerticalGap

\readerExcitement{喔~所以是「向量」加起來,再乘上一些數字?那可以給我一個具體的例子嗎?}

\begin{textRight}\begin{readerSaid}
如果有向量 $\vec{v}_1 = \begin{bmatrix}1 \\ 2\end{bmatrix}$ 和 $\vec{v}_2 = \begin{bmatrix}3 \\ -1\end{bmatrix}$,那它們的「線性組合」會是什麼?
\end{readerSaid}\end{textRight}

\begin{center}
中間的一些解釋...
\end{center}

\begin{textLeft}\begin{teacherSaid}
假設我們有兩個實數 $a$ 和 $b$,那麼 $\vec{v}_1$ 和 $\vec{v}_2$ 的線性組合就是:

\[
a \cdot \vec{v}_1 + b \cdot \vec{v}_2 = a \begin{bmatrix}1 \\ 2\end{bmatrix} + b \begin{bmatrix}3 \\ -1\end{bmatrix} = \begin{bmatrix}a + 3b \\ 2a - b\end{bmatrix}
\]

你可以想像,藉由改變 $a$ 和 $b$ 的值,我們就能在平面上得到很多不同的點。
\end{teacherSaid}\end{textLeft}

\teacherConfident{沒錯!這就是線性組合的精髓——用一組向量乘上數字,再加起來,去「生成」新的向量。這在電腦圖形、機器學習甚至物理中都非常重要!}

\smallerVerticalGap
\smallerVerticalGap

\readerEureka{原來如此!所以只要我有一組向量,我就可以用線性組合的方式去描述它們可以到達哪些地方!這樣的概念好像在搭積木呢!}


\end{document}
