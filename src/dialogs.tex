\usepackage{graphicx}     % 确保加载 graphicx 宏包
\usepackage{xeCJK}
\usepackage{framed}
\usepackage{ifthen}    % 載入 ifthen 套件
\usepackage{environ}

% ========== Character settings content ==========

\newcommand{\readerSettingsDescription}{
\begin{textRight}
\begin{minipage}[c]{\linewidth}
  {\Large \readerName} \\ \\
  {(He/Him) 一隻歐亞狼,對學習有高度的好奇心,遇到不同的概念總是第一個提問。跟 \teacherName 是的舊識好友,一起做思想實驗是他們常常做的娛樂活動。他會作為讀者的身分一起跟我們走過學習的旅程。}
\end{minipage}
\end{textRight}
}

\newcommand{\teacherSettingsDescription}{
\begin{textLeft}
\begin{minipage}[c]{\linewidth}
  {\Large \teacherName} \\ \\
  {(He/Him) 一隻龍型機器人 (類似於 Synth),內建核融合發電機,完全不需要吃東西也能存活。而且他非常聰明的他什麼都知道,聚集了毛類歷年以來的所有科技和知識結晶,沒有難得倒他的事情。是我們這次學習旅程的嚮導。}
\end{minipage}
\end{textLeft}
}

\newcommand{\readerSettings}{
  \begin{textAndImageEnv}
    {0.35\textwidth} % 左侧区域宽度
    {%
      \centering
      \includegraphics[width=\linewidth]{a-settings.png}
    }
    {1em}

    \textcolor{readerColor}{\readerSettingsDescription}
  \end{textAndImageEnv}
}

\newcommand{\teacherSettings}{
  \begin{imageAndTextEnv}
    {0.35\textwidth} % 左侧区域宽度
    {%
      \centering
      \includegraphics[width=\linewidth]{b-settings.png}
      % \vspace{0.5em}
      % \centering
      % \small
      % {\textcolor{YellowOrange}\readerName}
    }
    {1em}

    \textcolor{teacherColor}{\teacherSettingsDescription}
  \end{imageAndTextEnv}
}

\newcommand{\characterPage}{
\break
\section*{角色介紹}
\readerSettings
\teacherSettings
\break
}


% ========== Configs ==========

\graphicspath{ {./assets/} }

\setCJKmainfont{LXGW WenKai TC}
\setmainfont{LXGW WenKai TC}

% \setCJKmainfont{jf-openhuninn-2.0}
% \setmainfont{jf-openhuninn-2.0}

\newcommand{\readerName}{歐亞狼. 沃夫}
\newcommand{\teacherName}{IT. 龍}
\colorlet{readerColor}{YellowOrange}
\definecolor{teacherColor}{RGB}{21, 66, 130}

% ========== Gaps ==========

% --- 定義條件式的 vspace 指令 ---
% 如果文章段落很長,可以使用這種 Gap 來避免小文件文字被覆蓋的問題。
\newcommand{\smallerVerticalGap}{%
  % \lengthtest{} 用於比較長度
  % 檢查 \paperwidth 是否小於 180mm (A5=148mm, A4=210mm)
  \ifthenelse{\lengthtest{\paperwidth < 180mm}}{%
    % --- 如果為 TRUE (偵測到類 A5 尺寸) ---
    \par\vspace{0cm}% 在 A5 時使用的間距
    %\typeout{Paper detected as A5-ish, using vspace 0cm}% 取消註解可看訊息
  }{%
    % --- 如果為 FALSE (偵測到類 A4 或更大尺寸) ---
    \par\vspace{-2cm}% 在 A4 時使用的間距
    %\typeout{Paper detected as non-A5, using vspace -2cm}% 取消註解可看訊息
  }%
  % 加 \par 確保 vspace 在垂直模式下生效
}
% --- 結束定義 ---


% ========== Image & Text Composition Block ==========

% 定义主宏命令 (保持不变)
\newcommand{\imageAndText}[4]{%
  \par\noindent
  \begin{minipage}[c]{#1}%
    #2% <--- 图片或图片内容放这里
  \end{minipage}%
  \hspace{#3}%
  \begin{minipage}[c]{\dimexpr0.5\linewidth-#3\relax}%
    \begin{flushleft}
      #4% <--- 文本放这里
    \end{flushleft}
  \end{minipage}%
  \par
}

\newenvironment{imageAndTextEnv}[3]{
  \par\noindent
  \begin{minipage}[c]{#1}%
    #2% <--- 图片或图片内容放这里
  \end{minipage}%
  \hspace{#3}%
  \begin{minipage}[c]{\dimexpr0.5\linewidth-#3\relax}%
  \begin{flushleft}
}{
\end{flushleft}
\end{minipage}%
\par
}

\newcommand{\textAndImage}[4]{%
  \par\noindent
  \begin{flushright}
  \begin{minipage}[c]{\dimexpr0.5\linewidth-#3\relax}%
  \begin{flushright}
    #4% <--- 文本放这里
    \end{flushright}
  \end{minipage}%
  \hspace{#3}%
  \begin{minipage}[c]{#1}%
    #2% <--- 图片或图片内容放这里
  \end{minipage}%
  \end{flushright}
  \par
}

% ref: https://tex.stackexchange.com/questions/17036/why-cant-the-end-code-of-an-environment-contain-an-argument
\newenvironment{textAndImageEnv}[3]{
  \def\argNoI{#1}
  \def\argNoII{#2}
  \def\argNoIII{#3}

  \par\noindent
  \begin{flushright}
    \begin{minipage}[c]{\dimexpr0.5\linewidth-#3\relax}%
      \begin{flushright}
}
{
      \end{flushright}
    \end{minipage}%
    \hspace{\argNoIII}%
    \begin{minipage}[c]{\argNoI}%
      \argNoII % <--- 图片或图片内容放这里
    \end{minipage}%
  \end{flushright}
  \par
}

% ========== Positoin Block ==========

\newenvironment{textRight}[0]{%
  % Code executed at \begin{textRight}
  \par\noindent
  \begin{flushright}%
  \begin{minipage}[c]{\dimexpr0.75\textwidth\relax}%
}{%
  % Code executed at \end{textRight}
  \end{minipage}%
  \end{flushright}%
  \par%
}

\newenvironment{textLeft}[0]{%
  % Code executed at \begin{textRight}
  \par\noindent
  \begin{flushleft}%
  \begin{minipage}[c]{\dimexpr0.75\textwidth\relax}%
}{%
  % Code executed at \end{textRight}
  \end{minipage}%
  \end{flushleft}%
  \par%
}

% ========== Someone Said Paragraph Block ==========

\newenvironment{readerSaid}{
  \begin{textRight}
  \medskip
    \color{readerColor}
}{\medskip\end{textRight}}

\newenvironment{readerSaidFigure}{
  \begin{readerSaid}
}{\medskip\end{readerSaid}}

\newenvironment{teacherSaid}{
  \begin{textLeft}
  \medskip
    \color{teacherColor}
}{\medskip\end{textLeft}}

\newenvironment{teacherSaidFigure}{
  \begin{teacherSaid}
}{\medskip\end{teacherSaid}}



% ========== Reader Dialog ==========

% \newenvironment{readerDialogEnv}[1]{
%   \begin{textAndImageEnv}{0.25\textwidth}
%   {
%     \centering
%     \includegraphics[width=\linewidth]{#1}
%     \vspace{0.5em}
%     \centering
%     \small
%     {\textcolor{readerColor}\readerName}
%   }
%   {1em}
%   \setlength{\linewidth}{0.35\textwidth}
%   \color{teacherColor}
% }{
%   \end{textAndImageEnv}
% }

\newenvironment{readerDialogEnv}[1]{%
  \begin{textAndImageEnv}{0.25\textwidth}{%
    \centering
    \includegraphics[width=\linewidth]{#1}
    \vspace{0.5em}
    \centering
    \small
    {\textcolor{readerColor}{\readerName}}
  }{1em}

  \begin{minipage}[r]{\textwidth}
    \color{readerColor}
    \begin{flushright}
}{

\end{flushright}
\end{minipage}
  \end{textAndImageEnv}
}



\newenvironment{readerConfusedEnv}[0]{
\begin{readerDialogEnv}{a-confused.png}}
{\end{readerDialogEnv}}

\newcommand{\readerConfused}[1]{\begin{readerConfusedEnv} #1 \end{readerConfusedEnv}}

\newenvironment{readerEurekaEnv}[0]{
\begin{readerDialogEnv}{a-eureka.png}}
{\end{readerDialogEnv}}

\newcommand{\readerEureka}[1]{
\begin{readerEurekaEnv} #1 \end{readerEurekaEnv}}

\newenvironment{readerQuestioningEnv}[0]{
\begin{readerDialogEnv}{a-questioning.png}}
{\end{readerDialogEnv}}

\newcommand{\readerQuestioning}[1]{
\begin{readerQuestioningEnv} #1 \end{readerQuestioningEnv}}

\newenvironment{readerExcitementEnv}[0]{
\begin{readerDialogEnv}{a-excitement.png}}
{\end{readerDialogEnv}}

\newcommand{\readerExcitement}[1]{
\begin{readerExcitementEnv} #1 \end{readerExcitementEnv}}

\newenvironment{readerGrinningEnv}[0]{
\begin{readerDialogEnv}{a-grinning.png}}
{\end{readerDialogEnv}}

\newcommand{\readerGrinning}[1]{
\begin{readerGrinningEnv} #1 \end{readerGrinningEnv}}

\newenvironment{readerAgreeEnv}[0]{
\begin{readerDialogEnv}{a-agree.png}}
{\end{readerDialogEnv}}

\newcommand{\readerAgree}[1]{
\begin{readerAgreeEnv} #1 \end{readerAgreeEnv}}

\newenvironment{readerConcernEnv}[0]{
\begin{readerDialogEnv}{a-concern.png}}
{\end{readerDialogEnv}}

\newcommand{\readerConcern}[1]{
\begin{readerConcernEnv} #1 \end{readerConcernEnv}}

% ========== Teacher Dialog ==========

% \newenvironment{teacherDialogEnv}[1]{%
%   \begin{imageAndTextEnv}{0.25\textwidth}{%
%     \centering
%     \includegraphics[width=\linewidth]{#1}
%     \vspace{0.5em}
%     \centering
%     \small
%     {\textcolor{teacherColor}{\teacherName}}
%   }{%
%     1em
%   }%
%     \setlength{\linewidth}{0.35\textwidth}
%     \color{teacherColor}
% }{%
%   \end{imageAndTextEnv}
% }

\newenvironment{teacherDialogEnv}[1]{%
  \begin{imageAndTextEnv}{0.25\textwidth}{%
    \centering
    \includegraphics[width=\linewidth]{#1}
    \vspace{0.5em}
    \centering
    \small
    {\textcolor{teacherColor}{\teacherName}}
  }{1em}

  \begin{minipage}[r]{\textwidth}
    \color{teacherColor}
}{
\end{minipage}
  \end{imageAndTextEnv}
}


% Confident
\newenvironment{teacherConfidentEnv}{%
  \begin{teacherDialogEnv}{b-confident.png}%
}{%
  \end{teacherDialogEnv}%
}
\newcommand{\teacherConfident}[1]{%
  \begin{teacherConfidentEnv}#1\end{teacherConfidentEnv}%
}

% Talking
\newenvironment{teacherTalkingEnv}{%
  \begin{teacherDialogEnv}{b-talking.png}%
}{%
  \end{teacherDialogEnv}%
}
\newcommand{\teacherTalking}[1]{%
  \begin{teacherTalkingEnv}#1\end{teacherTalkingEnv}%
}

% Pause
\newenvironment{teacherPauseEnv}{%
  \begin{teacherDialogEnv}{b-pause.png}%
}{%
  \end{teacherDialogEnv}%
}
\newcommand{\teacherPause}[1]{%
  \begin{teacherPauseEnv}#1\end{teacherPauseEnv}%
}

% Agree
\newenvironment{teacherAgreeEnv}{%
  \begin{teacherDialogEnv}{b-agree.png}%
}{%
  \end{teacherDialogEnv}%
}
\newcommand{\teacherAgree}[1]{%
  \begin{teacherAgreeEnv}#1\end{teacherAgreeEnv}%
}

% Angry
\newenvironment{teacherAngryEnv}{%
  \begin{teacherDialogEnv}{b-angry.png}%
}{%
  \end{teacherDialogEnv}%
}
\newcommand{\teacherAngry}[1]{%
  \begin{teacherAngryEnv}#1\end{teacherAngryEnv}%
}

% Nervous
\newenvironment{teacherNervousEnv}{%
  \begin{teacherDialogEnv}{b-nervous.png}%
}{%
  \end{teacherDialogEnv}%
}
\newcommand{\teacherNervous}[1]{%
  \begin{teacherNervousEnv}#1\end{teacherNervousEnv}%
}

% Afraid
\newenvironment{teacherAfraidEnv}{%
  \begin{teacherDialogEnv}{b-afraid.png}%
}{%
  \end{teacherDialogEnv}%
}
\newcommand{\teacherAfraid}[1]{%
  \begin{teacherAfraidEnv}#1\end{teacherAfraidEnv}%
}
% ========== Plot color settings ==========

\newcommand{\teacherPlotSettings}{
  \pgfplotsset{
    axis line style={teacherColor},
    every axis label/.append style ={teacherColor},
    every tick label/.append style={teacherColor}
  }
}
